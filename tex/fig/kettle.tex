\begin{figure*}[t]
	\centering
	\addtolength{\tabcolsep}{-3pt}
	\begin{tabular}{ccc}
		\begin{overpic}[width=0.32\textwidth]{results/kettle_drop.jpg}
			\put(2,3){\bfseries \color{black} \Large (a)}
		\end{overpic}
		&
		\begin{overpic}[width=0.32\textwidth]{results/kettle_logo.jpg}
			\put(2,3){\bfseries \color{black} \Large (b)}
		\end{overpic}
		&
		\begin{overpic}[width=0.32\textwidth]{results/kettle_all.jpg}
			\put(2,3){\bfseries \color{black} \Large (c)}
		\end{overpic}
	\end{tabular}
	\caption{\label{fig:result_multilayer}
		\textbf{Multi-layer BSDF.}
		This result shows renderings of a kettle described with: 
		\textbf{(a)}~a single transparent layer with a dielectric top interface capturing the water drops over a conducting bottom surface with scratches; 
		\textbf{(b)}~a single translucent layer with spatially varying optical thicknesses and albedo over the same bottom surface of (a);
		\textbf{(c)}~a dual layer configuration created by stacking the transparent layer~(a) over the translucent one~(b).
	}
\end{figure*}